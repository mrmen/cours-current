\documentclass[12pt, french]{article}

\usepackage[utf8]{inputenc}
\usepackage[T1]{fontenc}
\usepackage{amsmath, amssymb, amsthm}
\usepackage{babel}
\usepackage{fancyhdr}
\usepackage[margin=2cm]{geometry}

\fancyhf{}
\fancyhead[L]{testTroisième}
\fancyhead[C]{Statistiques}
\fancyhead[R]{DS-Correction}
\pagestyle{fancy}

\begin{document}

\underline{\textbf{Méthode de notation}}

\begin{figure*}
\centering
\begin{tabular}{|*{5}{p{3cm}|}}
\hline
Positionnement & Insuffisant & Fragile & Satisfaisant & Très satisfaisant \\
\hline
Indicateur & Aucune réussite complète. & Seul un calcul parmi moyenne et médiane est exécuté correctement. & Les calculs de la moyenne et de la médiane sont justes. & Les calculs de la moyenne, de la médiane et la dernière question sont justes.\\	
\hline
\end{tabular}	
\caption{Compétence Représenter Domaine 1.3}
\end{figure*}

\begin{figure*}
\centering
\begin{tabular}{|*{3}{p{3cm}|}}
\hline
Positionnement & A revoir & OK \\
\hline
Indicuteur & La réponse ne comporte pas de nom de cellule et ne comporte pas de signe $=$ au départ. & La réponse contient des noms de cellules et/ou débute avec un $=$.\\
\hline
\end{tabular}	
\caption{Utilisation du tableur}
\end{figure*}

\underline{\textbf{Correction}}

\bigskip

	\begin{enumerate}
		\item Une formule dans un tableur débute toujours par un signe $=$. Il est possible d'obliger le tableur à réaliser la somme des effectifs avec l'expression suivante :
		\[= 8 + 2 + \cdots + 1 + 1\]
		mais la méthode la plus efficace est évidemment d'utiliser le nom des cellules pour rendre le calcul dynamique. La réponse attendue est donc la formule suivante :
		\[=B2 + C2 + \cdots + M2 + N2\]
		
		\item \begin{enumerate}
			\item Dans cet exercice, on peut utiliser la moyenne simple ou la moyenne pondérée. Pour rappel, les deux méthodes donnent \textbf{toujours} le même résultat.
			\begin{description}
				\item[moyenne simple]
				
				Il faut ici additionner toutes les valeurs du premier tableau puis diviser par le nombre de pays.
				\[moyenne = \dfrac{40 + 32 + \cdots + 1 + 1}{26}\]
				\item[moyenne pondérée]
				
				Il faut dans ce cas réaliser la somme des valeurs multipliées par leur effectif, et diviser le tout par l'effectif total.
				\[moyenne = \dfrac{1\times 8 + 2\times 2 + \cdots + 32\times 1 + 40\times 1}{26}\]
			\end{description}
			Dans les deux cas, le résultat arrondi à l'unité est $8$. On peut donc dire que le nombre moyen de médailles d'or gagnées est $8$.
			\item Pour déterminer la médiane, on regarde l'effectif total : ce dernier est pair, la médiane sera donc un nombre compris entre les deux valeurs centrales de la série statistiques ordonnée.

			$26 = 2 \times 13$, donc les deux valeurs centrales sont la $13\ieme$ et la $14\ieme$. On remarque à l'aide du tableau que ces deux valeurs sont égales $4$, donc la médiane est $4$.
			
			Ainsi, $50\%$ des pays au moins ont gagné plus de $4$ médailles d'or et au moins $50\%$ des pays ont gagné moins de $4$ médailles d'or.
			
			\item La moyenne est la valeur que devrait avoir chaque élément de la série statistique si ces derniers étaient tous égaux et que la moyenne était identique alors que la médiane est juste valeur qui sépare l'effectif en deux parties égales. De plus, les valeurs extrêmes d'une série statistique impliquent généralement cet effet.
		\end{enumerate}
	
	\item Si $70\%$	des pays ont gagné au moins une médaille d'or, alors ces pays sont les $26$ de l'exercice. On en déduit donc qu'il nous suffit de trouver combien représente $30\%$ pour avoir la réponse. On utilise donc la proportionnalité.
	\begin{center}
		\begin{tabular}{|c|c|c|}
		\hline
		Pourcentage & 70 & 30 \\
		\hline
		Nombre de pays & 26 & ? \\
		\hline
		\end{tabular}
	\end{center}
	Ainsi $30\%$ représente 
	\[26\times 30 \div 70 \simeq 11\]
	Onze pays environ n'ont obtenu que des médailles d'argent ou de bronze.
	\end{enumerate}

	
	
\end{document}